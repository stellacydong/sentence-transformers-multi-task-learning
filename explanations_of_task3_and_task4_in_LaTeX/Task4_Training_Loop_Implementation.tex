\documentclass{article}
\usepackage[utf8]{inputenc}
\usepackage{geometry}
\usepackage{booktabs}
\usepackage{listings}
\usepackage{xcolor}
\geometry{margin=1in}

\title{Task 4: Training Loop Implementation (BONUS)}
\author{ML Apprentice Take-Home Exercise}
\date{}

\lstset{
  language=Python,
  basicstyle=\ttfamily\small,
  keywordstyle=\color{blue},
  commentstyle=\color{gray},
  stringstyle=\color{orange},
  showstringspaces=false,
  breaklines=true,
  frame=single
}

\begin{document}

\maketitle

\section*{Introduction}
This document describes the design and structure of a multi-task training loop for the
\texttt{MultiTaskTransformer} model (sentence classification + NER). We focus on:
\begin{itemize}
  \item Handling of hypothetical batched data,
  \item The forward pass and combined loss computation,
  \item Metrics tracking for each task.
\end{itemize}
No actual training is executed here; the code is illustrative.

\section{Assumptions and Data Handling}
We assume a \texttt{DataLoader} yielding batches as dictionaries with:
\begin{itemize}
  \item \texttt{input\_ids}, \texttt{attention\_mask}: \texttt{Tensor} of shape $(B, L)$,
  \item \texttt{task\_a\_labels}: \texttt{Tensor} of shape $(B,)$ for sentence classes,
  \item \texttt{task\_b\_labels}: \texttt{Tensor} of shape $(B, L)$ for token tags,
        padded with \texttt{-100} to mask in loss.
\end{itemize}
A custom \texttt{collate\_fn} ensures correct stacking and padding:
\begin{lstlisting}
def collate_fn(batch):
    input_ids     = torch.stack([ex['input_ids'] for ex in batch])
    attention_mask= torch.stack([ex['attention_mask'] for ex in batch])
    task_a_labels = torch.tensor([ex['task_a_labels'] for ex in batch])
    task_b_labels = torch.stack([
        F.pad(torch.tensor(ex['task_b_labels']), (0, L - len(ex['task_b_labels'])), value=-100)
        for ex in batch
    ])
    return {
        'input_ids': input_ids,
        'attention_mask': attention_mask,
        'task_a_labels': task_a_labels,
        'task_b_labels': task_b_labels
    }
\end{lstlisting}

\section{Forward Pass and Loss Computation}
Each batch undergoes a single forward pass:
\begin{lstlisting}
# 1. Forward pass yielding two outputs:
logits_a, logits_b = model(input_ids, attention_mask)
# logits_a: (B, C_a), sentence classification
# logits_b: (B, L, C_b), token-level NER

# 2. Compute per-task losses:
loss_a = loss_fn_a(logits_a, task_a_labels)             # CrossEntropy on (B, C_a) vs (B,)
loss_b = loss_fn_b(
    logits_b.view(-1, C_b),                              # flatten to (B*L, C_b)
    task_b_labels.view(-1)                               # flatten to (B*L,)
)

# 3. Combined multi-task loss:
loss = loss_a + loss_b                                  # equal weighting
\end{lstlisting}
\noindent
Where \texttt{C\_a} is the number of sentence classes (e.g., 2) and \texttt{C\_b} is the number of NER tags (e.g., 4).

\section{Metrics Computation}
We track separate accuracy metrics for each task:

\begin{itemize}
  \item \textbf{Sentence Classification Accuracy (Task A):}
    \[
      \mathrm{Accuracy}_A 
      = \frac{1}{B} \sum_{i=1}^{B}
        \mathbf{1}\bigl(\arg\max(\mathrm{logits}_A^{(i)}) = y_A^{(i)}\bigr),
    \]
    where \(B\) is the batch size, \(\mathrm{logits}_A^{(i)}\) are the class scores for example \(i\), and \(y_A^{(i)}\) is its true label.
  
  \item \textbf{Token-Level NER Accuracy (Task B):}
    \[
      \mathrm{Accuracy}_B
      = \frac{\sum_{i=1}^{B}\sum_{j=1}^{L}
          \mathbf{1}\bigl(\hat y_{i,j} = y_{i,j}^B \bigr)\,m_{i,j}}
         {\sum_{i=1}^{B}\sum_{j=1}^{L} m_{i,j}},
    \]
    where:
    \begin{itemize}
      \item \(\hat y_{i,j} = \arg\max(\mathrm{logits}_B^{(i,j)})\) is the predicted tag for token \(j\) in example \(i\),
      \item \(y_{i,j}^B\) is the true tag (or \(-100\) if padded),
      \item \(m_{i,j} = \mathbf{1}(y_{i,j}^B \neq -100)\) masks out padding positions,
      \item \(L\) is the maximum sequence length.
    \end{itemize}
\end{itemize}


\section{Training Loop Pseudocode}
Putting it all together, a single epoch looks like:
\begin{lstlisting}
for epoch in range(num_epochs):
    model.train()
    total_loss = correct_a = total_a = correct_b = total_b = 0

    for batch in train_loader:
        optimizer.zero_grad()

        # Forward + loss
        logits_a, logits_b = model(batch['input_ids'], batch['attention_mask'])
        loss_a = loss_fn_a(logits_a, batch['task_a_labels'])
        loss_b = loss_fn_b(
            logits_b.view(-1, C_b), batch['task_b_labels'].view(-1)
        )
        loss = loss_a + loss_b

        # Backward & update
        loss.backward()
        optimizer.step()

        # Accumulate loss
        total_loss += loss.item()

        # Task A accuracy
        preds_a = logits_a.argmax(dim=1)
        correct_a += (preds_a == batch['task_a_labels']).sum().item()
        total_a   += batch['task_a_labels'].size(0)

        # Task B accuracy (mask padding)
        preds_b = logits_b.argmax(dim=2)
        mask    = batch['task_b_labels'] != -100
        correct_b += ((preds_b == batch['task_b_labels']) & mask).sum().item()
        total_b   += mask.sum().item()

    # Epoch metrics
    avg_loss = total_loss / len(train_loader)
    acc_a    = correct_a / total_a
    acc_b    = correct_b / total_b
    print(f"Epoch {epoch}: Loss={avg_loss:.4f}, "
          f"AccA={acc_a:.4f}, AccB={acc_b:.4f}")
\end{lstlisting}

\section*{Conclusion}
This multi-task training loop:
\begin{itemize}
  \item Efficiently handles batched data with custom collation.
  \item Executes a single shared forward pass for both tasks.
  \item Computes and aggregates separate metrics to monitor each task’s performance.
\end{itemize}
Joint backpropagation of both losses encourages the shared encoder to learn representations 
useful across tasks, improving overall generalization in a multi-task learning framework.

\end{document}
